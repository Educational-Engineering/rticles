\documentclass[10pt,paper=17cm:22cm, twoside=true, DIV=14]{scrbook}
\usepackage{scrhack}
\usepackage[ngerman]{babel}
% ============ Input Encoding ================

\usepackage[utf8]{inputenc}

\usepackage[T1]{fontenc}

% activate={true,nocompatibility} - activate protrusion and expansion
% final - enable microtype; use "draft" to disable
% tracking=true, kerning=true, spacing=true - activate these techniques
% factor=1100 - add 10% to the protrusion amount (default is 1000)
% stretch=10, shrink=10 - reduce stretchability/shrinkability (default is 20/20)
\usepackage[activate={true,nocompatibility},final,tracking=true,kerning=true,spacing=true,factor=1000,stretch=10,shrink=10]{microtype}

% ============= Avoid page breaks after paragraph and subsection =====================
\usepackage{xpatch}
\xpretocmd{\paragraph}{\needspace{1in}}{}{}
\xpretocmd{\subsubsection}{\needspace{1in}}{}{}
\usepackage{setspace}


% ============ Mathematik ================
\usepackage{amsmath}
\usepackage{amsfonts}
\usepackage{amssymb}

% ============ Tables ================
\usepackage{longtable}
\usepackage{booktabs}
\usepackage{array}

%more space between rows in Tables
\renewcommand{\arraystretch}{1.3}
\newcommand{\ra}[1]{\renewcommand{\arraystretch}{#1}}


% ============ Absaetze ================
\setlength{\parindent}{0pt}
\setlength{\parskip}{4pt}
\setkomafont{subparagraph}{\normalsize}



% ============ Bilder anpassen ================
\usepackage{graphicx}
\usepackage[export]{adjustbox}
\usepackage{subfigure}



% ============ Fonts ================
\usepackage{textcomp}



%% use the option 'defaultsans' instead of 'default' to replace the
%% sans serif font only.
\usepackage[defaultsans,osfigures,scale=1.0]{opensans} %% Alternatively



% ============ Colors & Boxes ================
\usepackage[framemethod=TikZ]{mdframed}
\usepackage{color}
\definecolor{taskColor}{RGB}{141,198,63} %green
\definecolor{infoColor}{RGB}{0,155,240} %blue
\definecolor{abstractColor}{RGB}{238,42,123} %red
\definecolor{orderColor}{RGB}{255,236,0} %yellow
\definecolor{gray}{RGB}{232,233,235}
\definecolor{grayLighter}{RGB}{241,242,242}

\mdfdefinestyle{task}{%
    linecolor=taskColor,
    outerlinewidth=1pt,
    roundcorner=10pt,
    innertopmargin=0.7\baselineskip,
    innerbottommargin=0.7\baselineskip,
    innerrightmargin=20pt,
    innerleftmargin=20pt,
    nobreak=true,
    backgroundcolor=gray}
\mdfdefinestyle{info}{%
    linecolor=infoColor,
    outerlinewidth=1pt,
    roundcorner=10pt,
    innertopmargin=0.7\baselineskip,
    innerbottommargin=0.7\baselineskip,
    innerrightmargin=20pt,
    innerleftmargin=20pt,
    nobreak=true,
    backgroundcolor=gray}
\mdfdefinestyle{abstract}{%
    linecolor=abstractColor,
    outerlinewidth=1pt,
    roundcorner=10pt,
    innertopmargin=0.7\baselineskip,
    innerbottommargin=0.7\baselineskip,
    innerrightmargin=20pt,
    innerleftmargin=20pt,
    nobreak=true,
    backgroundcolor=gray}
\mdfdefinestyle{order}{%
    linecolor=orderColor,
    outerlinewidth=1pt,
    roundcorner=10pt,
    innertopmargin=0.7\baselineskip,
    innerbottommargin=0.7\baselineskip,
    innerrightmargin=20pt,
    innerleftmargin=20pt,
    nobreak=true,
    backgroundcolor=gray}
\mdfdefinestyle{code}{
    topline=false,
    rightline=false,
    leftline=false,
    bottomline=false,
    roundcorner=10pt,
    innertopmargin=5pt,
    innerbottommargin=5pt,
    innerrightmargin=10pt,
    innerleftmargin=10pt,
    nobreak=true,
    backgroundcolor=grayLighter}

\newenvironment {task}
         {\begin{mdframed}[style=task] }
         {\end{mdframed} }

 \newenvironment {info}
          {\begin{mdframed}[style=info] }
          {\end{mdframed} }

\newenvironment {order}
         {\begin{mdframed}[style=order] }
         {\end{mdframed}}
\makeatletter
\@ifundefined{abstract}{
	\newenvironment {abstract}
        {\begin{mdframed}[style=abstract] }
         {\end{mdframed}}
}
{
	\renewenvironment {abstract}
        {\begin{mdframed}[style=abstract] }
         {\end{mdframed}}
}
\makeatother

% ============ Syntax Highlighting ================
  % commands and environments needed by pandoc snippets
     % extracted from the output of `pandoc -s`

\usepackage{lmodern}
\renewcommand{\ttdefault}{pcr}
\usepackage{fancyvrb}

\DefineVerbatimEnvironment{Highlighting}{Verbatim}{commandchars=\\\{\}}
          % Add ',fontsize=\small' for more characters per line
\newenvironment {Shaded}
        {\begin{mdframed}[style=code] }
         {\end{mdframed}}

\newcommand{\KeywordTok}[1]{\textcolor[rgb]{0.1,0.1,0.1}{\textbf{#1}}}
\newcommand{\BuiltInTok}[1]{\textcolor[rgb]{0.1,0.1,0.1}{\textbf{#1}}}
\newcommand{\DataTypeTok}[1]{\textcolor[rgb]{0.1,0.1,0.1}{\textbf{#1}}}
\newcommand{\DecValTok}[1]{\textcolor[rgb]{0.4,0.4,0.4}{{#1}}}
\newcommand{\BaseNTok}[1]{\textcolor[rgb]{0.1,0.1,0.1}{{#1}}}
\newcommand{\FloatTok}[1]{\textcolor[rgb]{0.4,0.4,0.4}{{#1}}}
\newcommand{\CharTok}[1]{\textcolor[rgb]{0.4,0.4,0.4}{{#1}}}
\newcommand{\StringTok}[1]{\textcolor[rgb]{0.4,0.4,0.4}{{#1}}}
\newcommand{\CommentTok}[1]{\textcolor[RGB]{128,128,128}{\textit{#1}}}
\newcommand{\OtherTok}[1]{#1}
\newcommand{\AlertTok}[1]{\textbf{#1}}
\newcommand{\FunctionTok}[1]{\textcolor[rgb]{0.2,0.2,0.2}{#1}}
\newcommand{\RegionMarkerTok}[1]{#1}
\newcommand{\ErrorTok}[1]{\textbf{#1}}
\newcommand{\NormalTok}[1]{#1}
\newcommand{\OperatorTok}[1]{{\textbf{#1}}}
\newcommand{\ControlFlowTok}[1]{{\textbf{#1}}}
\newcommand{\VariableTok}[1]{#1}




% ============ style inline code ===================
\BeforeBeginEnvironment{verbatim}{\begin{mdframed}[style=code]\shorthandoff{"}}
\AfterEndEnvironment{verbatim}{\shorthandon{"}\end{mdframed}\unskip}
\begingroup\lccode`~=`"
\lowercase{\endgroup
  \xapptocmd\ttfamily{\let~"}{}{}
}

% ============== E.Tutorial Setings ===============





% =========== books stuff ===================

\usepackage{scrlayer-scrpage}
\ohead{}
\cehead{}
\cohead{}

\newpairofpagestyles[scrheadings]{impressum}{\cfoot{\footnotesize  Version: 1, Datum: 30 August, 2016, Hash:  7d27443}}



%\renewcommand{\chaptermarkformat}{}
%\renewcommand{\sectionmarkformat}{}

% handle page in toc
\usepackage{tocstyle}
\usetocstyle{classic}

%\addto\captionsngerman{\renewcommand{\partname}{}} % remove part name

%reset counter after each part
\makeatletter
\@addtoreset{section}{part}
\makeatother
\renewcommand\thepart{\arabic{part}}
\setcounter{part}{-1}

%set chapter number
\renewcommand{\thechapter}{}
\renewcommand*\thesection{\thepart.\arabic{section}}
\renewcommand*\thetable{\thepart.\arabic{table}}
\renewcommand*\thefigure{\thepart.\arabic{figure}}
%center chapter
\addtokomafont{chapter}{\centering}

\providecommand{\tightlist}{%
  \setlength{\itemsep}{0pt}\setlength{\parskip}{0pt}}


\usepackage{multicol}
\usepackage{enumitem}


\title{Begleitunterlagen}
\author{
FirstName LastName \And Second Author
}



% ============ Links ================
\usepackage{hyperref}
\hypersetup{%
  colorlinks,
  linkcolor=black,
  citecolor=infoColor,
  urlcolor=infoColor
}

\begin{document}
\frontmatter
\begin{titlepage}
\begin{center}

\vspace*{3 cm} {\large \textsf{FirstName LastName, Second Author}}

\vspace{1.5 cm} {\huge \textsf{Programmieren mit Java}}

\vspace{0.5 cm} {\Huge \textbf{ \textsf{ Begleitunterlagen}}}\\
\vspace{0.5 cm} {\huge \textsf{Zu dem Online Kurs}}


\end{center}
\cleardoublepage
\par
\raisebox{0pt}{\includegraphics[width=0.25\textwidth]{figures/Logo.png}}%
\hfill
\raisebox{14pt}{\includegraphics[width=0.25\textwidth]{figures/teachITLogo.png}}%
\par

\begin{center}
\vspace{3 cm}
\noindent\rule{\textwidth}{0.4pt}


\vspace{0.5 cm} {\huge \textsf{Programmieren mit Java}}\\
\vspace{0.5 cm} {\Huge \textbf{ \textsf{ Begleitunterlagen}}}\\
\vspace{0.5 cm} {\huge \textsf{Zu dem Online Kurs}}

\noindent\rule{\textwidth}{0.4pt}

\vspace*{1.5 cm} {\large \textsf{FirstName LastName, Second Author}}

\clearpage


\thispagestyle{impressum}
\null
\vfill



\begin{center}
Trotz sorgfältiger Arbeit schleichen sich manchmal Fehler ein. Die Autoren sind Ihnen für Anregungen und Hinweise per Email an \href{mailto:et@ethz.ch}{\nolinkurl{et@ethz.ch}} dankbar!\\

\end{center}

\begin{center}
   Dieses Material steht unter der Creative-Commons-Lizenz\\
   \href{http://creativecommons.org/licenses/by-nc-nd/4.0/deed.de}{Namensnennung - Nicht kommerziell - Keine Bearbeitungen 4.0 International.} \\
    \end{center}
    \begin{center}
        \href{http://creativecommons.org/licenses/by-nc-nd/4.0/deed.de}{\includegraphics[scale=0.2]{figures/by-nc-nd_eu.png}}
    \end{center}
	\begin{center}
       Um eine Kopie dieser Lizenz zu sehen, besuchen Sie\\
       \href{http://creativecommons.org/licenses/by-nc-nd/4.0/deed.de}{http://creativecommons.org/licenses/by-nc-nd/4.0/deed.de}\\
	\end{center}
	Herstellung und Verlag: BoD – Books on Demand, Norderstedt\\
		  \vspace{0.3cm}
    {ISBN 978-3-527-70594-8}\\
	  \vspace{0.3cm}
\begin{minipage}{0.73\textwidth}
\begin{flushleft}
\begin{small}
  Bibliografische Information der Deutschen Nationalbibliothek\\
  Die Deutsche Nationalbibliothek verzeichnet diese Publikation in der Deutschen Nationalbibliografie; detaillierte bibliografische Daten sind im Internet über \href{http://dnb.dnb.de}{http://dnb.dnb.de} abrufbar.
\end{small}
\end{flushleft}
\end{minipage}
\end{center}
\end{titlepage}
\cleardoublepage

\setcounter{tocdepth}{2}
\tableofcontents

\mainmatter

\chapter{Wie soll dieses Buch verwendet werden?}

\begin{Shaded}
\begin{Highlighting}[]
\ControlFlowTok{if} \NormalTok{blub:}
  \BuiltInTok{print}\NormalTok{(}\StringTok{"blub"}\NormalTok{)}
\end{Highlighting}
\end{Shaded}

\begin{info}
\textbf{Einen Zellbereich bearbeiten}

Möchte man den Zellbereich A1 bis C3 bearbeiten, können zwei ineinander
verschachtelte Schleifen eingesetzt werden:

\begin{verbatim}
For x = 1 To 3
  For y = 1 To 3
    Cells(x,y)
  Next y
Next x
\end{verbatim}

\texttt{Cells} enthält als Zeilen- und Spaltenwerte die beiden
Laufvariablen. Dadurch werden sämtliche Kombinationen erzeugt:

\texttt{Cells(1,1)}, \texttt{Cells(1,2)}, \ldots{}, \texttt{Cells(3,3)}.
\end{info}

Das vorliegende Buch enthält alle Begleitunterlagen zum Onlinekurs
\textbf{Programmiergrundlagen mit Python und Matlab}. Für den Kurs
können Sie sich über eine der folgenden beiden Lehrveranstaltungen
registrieren und einschreiben:

\begin{itemize}
\tightlist
\item
  \textbf{Grundlagen der Informatik} (252-0852-00L),
  \emph{www.gdi.ethz.ch}
\item
  \textbf{Einsatz von Informatikmitteln} (252-0839-00L),
  \emph{www.evim.ethz.ch}
\end{itemize}

Der Kurs besteht aus folgenden \textbf{4 Modulen}:

\begin{table}[!htbp]\centering
\begin{tabular}{@{}llll@{}}
\toprule
 & Titel & Sprache \\
\midrule
 1 & Variablen und Datentypen & Python \\
 2 & Kontrollstrukturen und Logik & Python \\
 3 & Arrays, Simulieren und Modellieren & Python \\
 4 & Matrizenrechnen & Matlab \\
\bottomrule
\end{tabular}
\end{table}

Jedes Modul dauert abhängig von Ihrem Vorwissen 4 bis 8 Arbeitsstunden.
Die Materialien in diesem Buch und auf der Webseite begleiten Sie von
der Einführung der Begriffe und Konzepte, über deren Verwendung in
einfachen Programmier-Beispielen bis hin zur selbstständigen Anwendung
der Programmierkonzepte in kleinen Programmier-Projekten.

Jedes Modul ist wie folgt organisiert:

\begin{enumerate}
\def\labelenumi{\arabic{enumi}.}
\tightlist
\item
  \textbf{SEE}: In diesem Buch und in der Vorlesung erhalten Sie eine
  Einführung in die wichtigsten Begriffe und Konzepte der
  Programmierung.
\item
  \textbf{TRY}: In diesem Schritt wenden Sie die Konzepte erstmals
  anhand überschaubarer Programmierbeispielen an. Angeleitet werden Sie
  dabei von einem elektronischen Tutorials (E.Tutorial).
\item
  \textbf{DO}: In diesem Schritt setzten Sie selbstständig kleinere
  Programmier-Projekte um.
\item
  \textbf{EXPLAIN}: In diesem abschlissenden Schritt diskutieren Sie mit
  einer Assistenzperson im Computerraum über die Programmier-Projekte
  aus Phase 3. Dabei versuchen Sie, die neuen Konzepte der Phase 1
  anhand Ihrer Lösung aufzuzeigen.
\end{enumerate}

Dieses Buch enthält alle Begleitmaterialien für die Phasen 1 und 3.

\textbf{Danksagung:}

Wir danken folgenden Personen:

Prof.~Dr.~Hans Hinterberger und Dr.~Barbara Scheuner für das
Bereitstellen von Unterlagen und Aufgabenstellungen aus früheren Kursen.

Prof.~Dr.~Juraj Hromkovic für die finanzielle Unterstützung.

Dr.~Hans Joachim Böckenhauer für das Korrekturlesen.

\cleardoublepage \refstepcounter{part}
\addcontentsline{toc}{part}{\protect\numberline{\thepart} A Capitalized Title: Something about a Package der sehr lang ist }
\par \raisebox{0\height}{\includegraphics[width=0.25\textwidth]{figures/Logo.png}}
\hfill \raisebox{10pt}{\includegraphics[width=0.25\textwidth]{figures/teachITLogo.png} }
\par 

\begin{center} \noindent\rule{\textwidth}{0.4pt}\\ \vspace{0.5 cm} {\large \textsf{ A Capitalized Title }\par} \vspace{0.5 cm} {\LARGE \textbf{\textsf{ A Capitalized Title: Something about a Package der sehr lang ist  }}\par} \vspace{0.5 cm} {\Large \textsf{ Theorieteil }\par} \noindent\rule{\textwidth}{0.4pt}\\ \vspace{1 cm}Autoren: \\ \vspace{2 mm} FirstName LastName, Second Author \\ \end{center}

\section*{Begriffe}

\begin{minipage}{\textwidth} \hrule \begin{multicols}{2} \begin{itemize}[leftmargin=0mm] \item[] keywords \item[] not capitalized \item[] Java \end{itemize} \end{multicols} \vspace{-2mm} \hrule \end{minipage}

\clearpage \begingroup \let\clearpage\relax \let\cleardoublepage\relax \chapter{ Theorieteil } \endgroup 

\section{Modulübersicht Test 2}\label{modulubersicht-test-2}

Die beiden Konzepte Variablen und Datentypen sind für jede
Programmierung grundlegend. Bei \textbf{Variablen} handelt es sich um
Speicherbereiche, in denen Werte gespeichert werden können und der
\textbf{Datentyp} gibt an, welche Werte erlaubt sind (z.B. nur
Ganzzahlen). In einem Programm werden Daten verarbeitet, die sich in
ihrer Art unterscheiden, z.B. Zeichen, Zahlen oder logische Daten.
Digitale Daten werden immer durch Ziffern dargestellt. Daher auch der
Name, \emph{digit} bedeutet Ziffer.

\section{Darstellung von Zahlen und Zeichen im
Computer}\label{darstellung-von-zahlen-und-zeichen-im-computer}

Um die Darstellung von Zeichen, Zahlen und Texten im Computer zu
verstehen, muss man das \textbf{binäre System} verstehen.

\subsection{Binäres System}\label{binares-system}

Alle Rechner stellen Information im binären System dar. Dieses kennt nur
zwei Ziffern, nämlich 0 und 1 (im Gegensatz zum Dezimalsystem mit den
Ziffern 0 bis 9). Eine solche Ziffer wird als \textbf{Bit} bezeichnet
(Abkürzung für \emph{Binary Digit}, übersetzt „Binäre Ziffer``). Ein Bit
stellt den kleinsten speicherbaren Wert in einem Computer dar. Jeweils 8
Bits werden zu einem \textbf{Byte} zusammengefasst. Ein Byte kann somit
2\textsuperscript{8} = 256 verschiedene Sequenzen von je 8 Bit
speichern.

\subsection{Darstellung von Zahlen im binären
System}\label{darstellung-von-zahlen-im-binaren-system}

Betrachten wir die Zahl 91, die binär mit 8 Bit als 01011011 dargestellt
wird (siehe Tabelle \ref{tab:binaryConversation}). Wir reden deswegen in
diesem Zusammenhang von der \textbf{Binärdarstellung} von 91 (und nicht
von der Dezimaldarstellung, die für uns lesefreundlicher ist).

\begin{table}[!htpb]
\begin{tabular}{|l|p{.8cm}|p{.8cm}|p{.8cm}|p{.8cm}|p{.7cm}|p{.7cm}|p{.7cm}|p{.7cm}|p{.8cm}}
\cline{1-9} \textbf{Bit} & 8 & 7 & 6 & 5 & 4 & 3 & 3 & 1 &\\
\cline{1-9} \textbf{Binärwert} & 0 & 1 & 0 & 1 & 1 & 0 & 1 & 1 &\\
\cline{1-9} \textbf{Wertigkeit} & $2^7=128$ & $2^6=64$ & $2^5=32$ & $2^4=16$ & $2^3=8$ & $2^2=4$ & $2^1=2$ & $2^0=1$ &\\
\hline \textbf{Dezimalwert} & 0 & 64 & 0 & 16 & 8 & 0 & 2 & 1 & \multicolumn{1}{ p{.8cm}| }{= 91} \\
\hline
\end{tabular}
\caption{Binäre Darstellung der Dezimalzahl 91. Details siehe Text.}
\label{tab:binaryConversation}
\end{table}

Eine 8-Bit-Zahl, wie in unserem Beispiel, kann Werte zwischen 00000000
(0 im Dezimalsystem) und 11111111 (255 im Dezimalsystem) speichern. Für
die Umrechnung vom Binär- in den Dezimalwert multiplizieren wir für
jedes Bit den Binärwert mit der Wertigkeit des Bits und summieren diese
auf. Im binären System können wir mit 8 Bit nur die ganzen Zahlen 0 bis
255 darstellen. Ist die Zahl, die wir darstellen wollen, grösser, muss
auch ein grösserer Speicherbereich bereitgestellt werden.

\subsection{Darstellung von Zeichen im binären
System}\label{darstellung-von-zeichen-im-binaren-system}

Für die Darstellung von Zeichen im Computer wurde der so genannte
\textbf{ASCII-Code} entwickelt. ASCII steht für \emph{American Standard
Code for Information Interchange}, was übersetzt so viel heisst wie
Amerikanische Standardcodierung für den Datenaustausch. Mit Hilfe des
7-Bit-ASCII-Codes können 128 verschiedene Zeichen (2\textsuperscript{7})
dargestellt werden oder umgekehrt wird jedem Zeichen ein Bitmuster aus 7
Bit zugeordnet (siehe Tabelle \ref{tab:ascii}). Die Zeichen entsprechen
weitgehend einer Computertastatur. Der ASCII-Code wurde später auf 8 Bit
erweitert, was die Darstellung von 256 Zeichen (2\textsuperscript{8})
erlaubt.

\begin{table}[!htbp]\centering
\ra{1.1}
\begin{tabular}{@{}llcllcllcll@{}}
\toprule
\multicolumn{2}{c}{0-31} & & \multicolumn{2}{c}{31-63} & &
\multicolumn{2}{c}{64-95} & & \multicolumn{2}{c}{96-127}\\
\cmidrule{1-2} \cmidrule{4-5} \cmidrule{7-8} \cmidrule{10-11}
Dez & Zeichen & & Dez  & Zeichen & & Dez & Zeichen & & Dez  & Zeichen\\
\midrule
0  & NUL & & 32 &  SP & & 64 & @ & & 96  & ` \\
1  & SOH & & 33 &  !  & & 65 & A & & 97  & a \\
2  & STX & & 34 &  "' & & 66 & B & & 98  & b \\
3  & ETX & & 35 &  \# & & 67 & C & & 99  & c \\
4  & EOT & & 36 &  \$ & & 68 & D & & 100 & d \\
5  & ENQ & & 37 &  \% & & 69 & E & & 101 & e \\
6  & ACK & & 38 &  \& & & 70 & F & & 102 & f \\
7  & BEL & & 39 &  '  & & 71 & G & & 103 & g \\
8  & BS  & & 40 &  (  & & 72 & H & & 104 & h \\
9  & HT  & & 41 &  )  & & 73 & I & & 105 & i \\
10 & LF  & & 42 &  *  & & 74 & J & & 106 & j \\
11 & VT  & & 43 &  +  & & 75 & K & & 107 & k \\
12 & FF  & & 44 &  ,  & & 76 & L & & 108 & l \\
13 & CR  & & 45 &  -  & & 77 & M & & 109 & m \\
14 & SO  & & 46 &  .  & & 78 & N & & 110 & n \\
15 & SI  & & 47 &  /  & & 79 & O & & 111 & o \\
16 & DLE & & 48 &  0  & & 80 & P & & 112 & p \\
17 & DC1 & & 49 &  1  & & 81 & Q & & 113 & q \\
18 & DC2 & & 50 &  2  & & 82 & R & & 114 & r \\
19 & DC3 & & 51 &  3  & & 83 & S & & 115 & s \\
20 & DC4 & & 52 &  4  & & 84 & T & & 116 & t \\
21 & NAK & & 53 &  5  & & 85 & U & & 117 & u \\
22 & SYN & & 54 &  6  & & 86 & V & & 118 & v \\
23 & ETB & & 55 &  7  & & 87 & W & & 119 & w \\
24 & CAN & & 56 &  8  & & 88 & X & & 120 & x \\
25 & EM  & & 57 &  9  & & 89 & Y & & 121 & y \\
26 & SUB & & 58 &  :  & & 90 & Z & & 122 & z \\
27 & ESC & & 59 &  ;  & & 91 & [ & & 123 & \{\\
28 & FS  & & 60 &  <  & & 92 & $\backslash$ & & 124 & $\mid$\\
29 & GS  & & 61 &  =  & & 93 & ]    & & 125 & \}  \\
30 & RS  & & 62 &  >  & & 94 & \^{} & & 126 & \textasciitilde \\
31 & US  & & 63 &  ?  & & 95 & \_   & & 127 & DEL \\
\bottomrule
\end{tabular}
\caption {ASCII-Tabelle}
\label{tab:ascii}
\end{table}

Die ASCII-Tabelle enthält auch nicht darstellbare Zeichen (wie etwa ein
Zeichen, das einen Zeilenumbruch repräsentiert). Die wichtigsten sind in
Tabelle \ref{tab:asciiHidden} dargestellt:

\begin{table}[!htbp]\centering
\begin{tabular}{lll}
\toprule
Dez & Zeichen & Bedeutung\\
\midrule
8   & BS & Backspace. Linkes Zeichen löschen\\
10  & NL & New Line. Neue Zeile beginnen\\
32  & SP & Space. Leerzeichen \\
127 & DEL & Delete. Rechtes Zeichen löschen \\
\bottomrule
\end{tabular}
\label{tab:asciiHidden}
\caption{Nicht darstellbare Zeichen der ASCII-Tabelle}
\end{table}

\section{Datentypen}\label{datentypen}

Der \textbf{Datentyp} gibt an, welche Daten in einem Programm
gespeichert und bearbeitet werden können. Programmiersprachen besitzen
vordefinierte Datentypen, die sich in der Art der Interpretation der
gespeicherten Daten und in der Grösse unterscheiden.

\begin{itemize}
\tightlist
\item
  Typ für Zahlenwerte
\item
  Typ für Zeichenwerte
\item
  Typ für Wahrheitswerte (Boolsche Werte) (siehe Modul 2)
\end{itemize}

Tabelle \ref{tab:datatypes} gibt einen Überblick über die wichtigsten
Datentypen, die in vielen Programmiersprachen vorkommen.

\begin{table}[!htbp]\centering 
\begin{tabular}{@{}llll@{}}     
\toprule
 Typ & Beschreibung & Grösse & Wertebereich \\
 & & in Bit & \\
\midrule
 boolean & Boolscher Wert & 1 & true oder false \\
 char & Zeichen & 16 & Unicode-Zeichen \\
 byte & Ganzzahl & 8 & $-128 \ldots 127 \;(-2^7 \ldots +2^{7}-1)$ \\
 short & & 16 & $-32'768 \ldots  32'767 \;(-2^{15} \ldots +2^{15}-1)$ \\
 int & & 32 & $-2'147'483'648 \ldots2'147'483'647 \;(-2^{31} \ldots +2^{31}-1)$ \\
 long & & 64 & $-9'223'372'036'854'775'808\ldots$ \\
 & & &  $9'223'372'036'854'775'807 \; (-2^{63} \ldots +2^{63}-1)$ \\
 float & Gleitkommazahl & 32 & $+/- 3.40282347 \times 10^{38}$\\
 double & & 64 & $+/- 1.79769313486231569 \times 10^{308}$ \\
\bottomrule

\end{tabular}
\caption{Die wichtigsten Datentypen in Java}
\label{tab:datatypes}
\end{table}

\section{Variablen und Konstanten}\label{variablen-und-konstanten}

\textbf{Variablen} einer Programmiersprache sind denen der Mathematik
ähnlich. Variablen haben einen Namen über den sie angesprochen werden,
und dienen dazu, einen Wert im Bereich eines bestimmten Datentyps zu
speichern. Der Wert der Variable kann sich im Verlaufe des Programms
ändern (er kann variieren, daher der Name). Um eine Variable in einem
Programm verwenden zu können, sind folgende Operationen notwendig:

\begin{enumerate}
\def\labelenumi{\arabic{enumi}.}
\tightlist
\item
  Deklaration
\item
  Initialisierung
\end{enumerate}

\subsection{Deklaration}\label{deklaration}

Bevor eine Variable in einem Programm verwendet werden kann, muss sie
\textbf{deklariert} werden. Das heisst, dass Sie als Programmiererin
oder Programmierer einen Speicherbereich für einen bestimmten Datentyp
belegen und diesem Speicherplatz einen Namen geben. Über diesen Namen
kann der Speicherbereich während des Programmablaufs aufgerufen werden.
Namen von Variablen beginnen in Java gemäss Konvention jeweils mit einem
Kleinbuchstaben, sie dürfen keine Leerzeichen enthalten und sollten
möglichst aussagekräftig sein.

\paragraph{Schreibweise:}\label{schreibweise}

\begin{verbatim}
Datentyp name;
\end{verbatim}

\paragraph{Beispiel:}\label{beispiel}

\begin{Shaded}
\begin{Highlighting}[]
\CommentTok{// Variable a vom Typ Integer}
\DataTypeTok{int} \NormalTok{a;}

\CommentTok{// Variable b vom Typ Double}
\DataTypeTok{double} \NormalTok{b;}

\CommentTok{// Variable c vom Typ Character}
\DataTypeTok{char} \NormalTok{c;}
\end{Highlighting}
\end{Shaded}

Mehrere Variablen vom gleichen Typ können auch wie in folgendem Beispiel
in einer Deklaration geschrieben werden:

\begin{Shaded}
\begin{Highlighting}[]
\CommentTok{// 3 Variablem vom Typ Integer}
\DataTypeTok{int} \NormalTok{meineZahl1, meineZahl2, meineZahl3;}
\end{Highlighting}
\end{Shaded}

\subsection{Initialisierung und
Wertzuweisung}\label{initialisierung-und-wertzuweisung}

Das \textbf{Zuweisen} von Werten geschieht mit dem Zuweisungsoperator.
In Java wird hierfür ein \textbf{Gleichheitszeichen} (\textbf{=})
verwendet. Dabei wird der Wert des Ausdrucks rechts des
Zuweisungsoperators der Variablen auf der linken Seite zugewiesen. Wenn
einer Variable das erste Mal ein Wert zugewiesen wird, spricht man von
einer \textbf{Initialisierung}.

\paragraph{Schreibweise:}\label{schreibweise-1}

\begin{Shaded}
\begin{Highlighting}[]
\NormalTok{variable = wert;}
\end{Highlighting}
\end{Shaded}

\paragraph{Beispiel:}\label{beispiel-1}

\begin{Shaded}
\begin{Highlighting}[]
\NormalTok{meineZahl = }\DecValTok{4}\NormalTok{;}
\CommentTok{// meineZahl hat den Wert 4}
\end{Highlighting}
\end{Shaded}

Der Wert einer Variablen kann sich im Verlaufe eines Programms ändern.
In folgendem Beispiel wird in der Variablen \texttt{meineZahl} zuerst
der Wert 4 gespeichert, der dann in einer weiteren Zeile mit dem Wert 6
überchrieben wird.

\begin{Shaded}
\begin{Highlighting}[]
\NormalTok{meineZahl = }\DecValTok{4}\NormalTok{;}
\CommentTok{// Wert von meineZahl ist 4}

\NormalTok{meineZahl = }\DecValTok{6}\NormalTok{;}
\CommentTok{// Wert von meineZahl ist 6}
\end{Highlighting}
\end{Shaded}

Bei einer Zuweisung handelt es sich also immer um einen schreibenden
Zugriff auf eine Variable mit dem Resultat, dass sich deren Wert ändern
kann. Der alte Wert wird überschrieben. Damit einer Variablen ein Wert
zugewiesen werden kann, darf die Variable nicht als Konstante definiert
sein (siehe nächster Abschnitt) und der Typ der Variablen muss mit dem
Typ des Werts kompatibel sein. Auf jeden Fall kompatibel sind Variablen
und Werte desselben Datentypes. Wenn die Datentypen nicht
übereinstimmen, nimmt Java eine implizite \textbf{Typkonvertierung} vor.
Dies ist jedoch eine häufige Fehlerquelle und sollte daher vermieden
werden. Bei der impliziten Typkonvertierung in Java werden nur
Typkonvertierung durchgeführt, wenn sie ohne Informationsverlust
erfolgen kann, also wenn der Zieldatentyp einen gleichen oder grösseren
Wertebereich hat als der Ausgangsdatentyp.

\paragraph{Beispiel:}\label{beispiel-2}

Ein Wert vom Typ \texttt{int} kann einer Variablen vom Typ
\texttt{double} zugewiesen werden:

\begin{Shaded}
\begin{Highlighting}[]
\NormalTok{# rear}
\CommentTok{// Variable ganzeZahl vom Typ Integer}
\DataTypeTok{int} \NormalTok{ganzeZahl;}
\CommentTok{// Variable kommaZahl vom Typ Double}
\DataTypeTok{double} \NormalTok{kommaZahl;}

\NormalTok{ganzeZahl = }\DecValTok{4}\NormalTok{;}
\NormalTok{kommaZahl = ganzeZahl;}
\CommentTok{// Variable kommaZahl wird in zum Typ Integer konvertiert}
\end{Highlighting}
\end{Shaded}

Möchte man eine Zuweisung machen, bei der der Zieldatentyp einen
gleineren Wertebereich hat, muss eine so genannte explizite
Typenkonvertierung durchgeführt werden, das sogenannte Typecasting. Die
Programmiererin/der Programmierer ist dabei selber dafür verantwortlich,
dass die Zuweisung möglich ist.

\paragraph{Beispiel:}\label{beispiel-3}

\begin{Shaded}
\begin{Highlighting}[]
\CommentTok{// Variable d vom Typ Double}
\DataTypeTok{double} \NormalTok{d = }\FloatTok{1.3}\NormalTok{;}

\DataTypeTok{float} \NormalTok{f = (}\DataTypeTok{float}\NormalTok{)d;}
\CommentTok{// Variable d muss explizit zum Typ Float konvertiert }
\CommentTok{// werden}
\end{Highlighting}
\end{Shaded}

Eine Variable kann in einem Programm nur in einem bestimmten Bereich des
Programms gelten. Weiteres dazu erfahren Sie in einem späteren Modul.

\subsection{Konstanten}\label{konstanten}

\textbf{Konstanten} werden wie Variablen mit einem Namen bezeichnet. Sie
enthalten während der gesamten Programmausführung einen konstanten Wert.
Es kann also nach der Initialisierung keine weitere Wertzuweisung
vorgenommen werden. Konstanten können jedoch Teil einer Wertzuweisung an
Variablen sein. Eine Konstante wird zusätzlich zu Namen und Datentyp mit
dem Schlüsselwort \texttt{final} deklariert.

\paragraph{Schreibweise:}\label{schreibweise-2}

\begin{Shaded}
\begin{Highlighting}[]
\DataTypeTok{final} \NormalTok{Datentyp name;}
\end{Highlighting}
\end{Shaded}

\paragraph{Beispiel:}\label{beispiel-4}

\begin{Shaded}
\begin{Highlighting}[]
\CommentTok{//Deklaration und Initialisierung der Konstante k}
\DataTypeTok{final} \DataTypeTok{int} \NormalTok{k = }\DecValTok{4}\NormalTok{;}
\end{Highlighting}
\end{Shaded}

\section{Operatoren und Ausdrücke}\label{operatoren-und-ausdrucke}

\subsection{Operatoren (Teil I)}\label{operatoren-teil-i}

Um in einem Programm Berechnungen durchzuführen zu können, stehen
folgende \textbf{arithmetische Operatoren} zur Verfügung:

\begin{table}[!htpb]\centering
\begin{tabular}{@{}llll@{}}

\toprule
 Operator & Ausdruck & Beschreibung & Liefert \\
\midrule
 + & a + b & Addition & Summe \\
 - & a - b& Subtraktion & Differenz \\
 $\ast$ & a $\ast$ b& Multiplikation& Produkt \\
 / & a / b& Division& Quotient \\
 \% & a \% b& Modulo& Ganzzahliger Rest einer Division \\
\bottomrule
\end{tabular}
\caption{Arithmetische Operatoren in Java}
\label{tab:Operators1}
\end{table}

Weitere Operatoren (logische und Vergleichsoperatoren) lernen Sie in
Modul 2 kennen.

\subsection{Ausdrücke}\label{ausdrucke}

\textbf{Ausdrücke} (engl. \emph{expressions}) sind in einer
Programmiersprache Teil der kleinsten ausführbaren Einheiten eines
Programms. Dabei handelt es sich um Verarbeitungsvorschriften, die sich
aus \textbf{Variablen}, \textbf{Konstanten} und \textbf{Operatoren}
zusammensetzen können und ein Resultat ergeben. Variablen und
Konstanten, die mit einem Operator verknüpft werden, nennt man
\textbf{Operanden}. Ein Ausdruck kann auch aus einer einzelnen Variablen
bestehen.

Folgendes Beispiel zeigt einen Ausdruck, der aus einer Variablen
\texttt{i}, einem Operator \texttt{+} und einem Operanden \texttt{5}
besteht.

\begin{Shaded}
\begin{Highlighting}[]
  \NormalTok{i + }\DecValTok{5}
\end{Highlighting}
\end{Shaded}

Das Resultat des Ausdrucks kann wieder in einer Variablen gespeichert
werden. In folgendem Beispiel wird das Resultat in der Variablen
\texttt{i} gespeichert. Der vorherige Wert von \texttt{i} wird dadurch
überschrieben.

\begin{Shaded}
\begin{Highlighting}[]
  \NormalTok{i = i + }\DecValTok{5}\NormalTok{;}
\end{Highlighting}
\end{Shaded}

Die Reihenfolge, mit der Ausdrücke bearbeitet werden, kann durch die
Wahl des Operators und durch Klammern beeinflusst werden. Hierfür gelten
die mathematischen Regeln, wie wir sie in der Schule gelernt haben, also
„Klammern zuerst, dann Punkt vor Strich``.

\paragraph{Beispiel:}\label{beispiel-5}

\begin{Shaded}
\begin{Highlighting}[]
  \DecValTok{5} \NormalTok{* (}\DecValTok{2} \NormalTok{+ }\DecValTok{10}\NormalTok{)}
\end{Highlighting}
\end{Shaded}

Die Klammern erzwingen, dass die Addition vor der Multiplikation
ausgeführt wird.

\subsection{Weitere Arithmetische
Operatoren}\label{weitere-arithmetische-operatoren}

Es gibt in Java noch weitere arithmetische Operatoren.

\paragraph{Zuweisungsoperator:}\label{zuweisungsoperator}

\begin{Shaded}
\begin{Highlighting}[]
\NormalTok{i += }\DecValTok{1}\NormalTok{; \textbackslash{}\textbackslash{}entspricht i = i + }\DecValTok{1}\NormalTok{;}
\NormalTok{i -= }\DecValTok{1}\NormalTok{; \textbackslash{}\textbackslash{}entspricht i = i - }\DecValTok{1}\NormalTok{;}
\NormalTok{i *= }\DecValTok{1}\NormalTok{; \textbackslash{}\textbackslash{}entspricht i = i * }\DecValTok{1}\NormalTok{;}
\NormalTok{i /= }\DecValTok{1}\NormalTok{; \textbackslash{}\textbackslash{}entspricht i = i / }\DecValTok{1}\NormalTok{;}
\NormalTok{i %= }\DecValTok{1}\NormalTok{; \textbackslash{}\textbackslash{}entspricht i = i % }\DecValTok{1}\NormalTok{;}
\end{Highlighting}
\end{Shaded}

Die Zuweisungsoperatoren dienen dazu die Anweisungen kompakter
darzustellen, da man weniger Zeichen benötigt.

\paragraph{Increment und Decrement
Operatoren:}\label{increment-und-decrement-operatoren}

\begin{Shaded}
\begin{Highlighting}[]
\NormalTok{i++; \textbackslash{}\textbackslash{}entspricht i = i + }\DecValTok{1}\NormalTok{;}
\NormalTok{i--; \textbackslash{}\textbackslash{}entspricht i = i - }\DecValTok{1}\NormalTok{;}
\end{Highlighting}
\end{Shaded}

Diese Operatoren sind meistens in sogenannten for-Schleifen anzutreffen,
wo sie einen Zähler hochzählen (siehe Modul 2) .

\section{Der Datentyp String}\label{der-datentyp-string}

Der Datentyp \textbf{String} unterscheidet sich von den bisher
thematisierten Datentypen insofern, dass er eine Zusammenfassung von
mehreren gleichartigen Variablen darstellt. Dieser Datentyp ist auch
kein primitiver Datentyp mehr, da er mehrere Elemente zusammenfasst. Er
speichert nämlich alle Buchstaben einzeln in je einer char-Variablen.
Wie diese Zusammenfassung der einzelnen Buchstaben funktioniert, lernen
Sie, wenn es um die Objektorientierung geht. Die Deklaration und
Initialisierung der Variablen funktioniert jedoch wie in 1.4.1 und 1.4.2
beschreiben.

Bei der Initialisierung von String-Variablen muss der Wert zwischen
\textbf{doppelten Hochkommata} (``) angegeben werden.

\paragraph{Beispiel:}\label{beispiel-6}

\begin{Shaded}
\begin{Highlighting}[]
\CommentTok{// Deklaration des Strings vorname}
\NormalTok{String vorname;}

\CommentTok{// Initialisierung mit dem Wert "Paul"}
\NormalTok{vorname = }\StringTok{"Paul"}\NormalTok{;}
\end{Highlighting}
\end{Shaded}

Da ein String mehrere char-Variablen enthält, kann dem String auch ein
einzelner char zugewiesen werden.

\paragraph{Beispiel:}\label{beispiel-7}

\begin{Shaded}
\begin{Highlighting}[]
\NormalTok{String name;}
\NormalTok{name = 'a';}
\end{Highlighting}
\end{Shaded}

Mehrere Strings können mit einem \textbf{Plus} (+) verbunden werden. So
entsteht aus mehreren Einzelteilen ein neuer Text.

\paragraph{Beispiel:}\label{beispiel-8}

\begin{Shaded}
\begin{Highlighting}[]
\NormalTok{String text;}
\NormalTok{text = }\StringTok{"Hallo, "}\NormalTok{+ }\StringTok{"das "} \NormalTok{+ }\StringTok{"sind "} \NormalTok{+ }\StringTok{"mehrere "} \NormalTok{+ }\StringTok{"Wörter."}\NormalTok{;}
\end{Highlighting}
\end{Shaded}

\section{Bildschirm- Ein- und
Ausgabe}\label{bildschirm--ein--und-ausgabe}

Oft möchte man, dass die Benutzerin oder der Benutzer des Programms mit
dem Programm interagieren kann. Das bedeutet, dass das Programm eine
Ausgabe macht oder dass die Benutzerin oder der Benutzer etwas eingeben
kann. Um dies zu realisieren verwenden wir Funktionalitäten, welche von
Java zur Verfügung gestellt werden.

\subsection{Ausgabe}\label{ausgabe}

Damit die Benutzerin oder der Benutzer auch sieht, was im Programm
berechnet wurde, kann im Programmcode angegeben werden, dass ein
bestimmter Text oder der Wert einer Variablen ausgegeben wird.

\paragraph{Beispiel: Ausgabe eines vorgegebenen
Texts}\label{beispiel-ausgabe-eines-vorgegebenen-texts}

\begin{Shaded}
\begin{Highlighting}[]
\NormalTok{System.}\FunctionTok{out}\NormalTok{.}\FunctionTok{println}\NormalTok{(}\StringTok{"Das Programm hat geendet."}\NormalTok{);}
\end{Highlighting}
\end{Shaded}

Im obigen Beispiel wird der Text „Das Programm hat geendet.''
ausgegeben. Der Text, der ausgegeben wird, steht zwischen einem Paar von
Anführungs- und Schlusszeichen (``), die nicht mit ausgegeben werden.
Man möchte aber nicht immer nur vorgegebenen Text ausgeben, sondern z.B.
das Resultat einer Berechnung, welches in einer Variablen (z.B.
\texttt{ganzeZahl}) gespeichert ist.

\paragraph{Beispiel: Ausgabe des Wertes einer
Variablen}\label{beispiel-ausgabe-des-wertes-einer-variablen}

\begin{Shaded}
\begin{Highlighting}[]
\NormalTok{System.}\FunctionTok{out}\NormalTok{.}\FunctionTok{println}\NormalTok{(ganzeZahl);}
\end{Highlighting}
\end{Shaded}

Wenn man Variablenwerte und Text verbinden möchte, geschieht dies mit
einem Additions-Zeichen (\texttt{+}).

\paragraph{Beispiel: Ausgabe von Text und
Variablenwert}\label{beispiel-ausgabe-von-text-und-variablenwert}

\begin{Shaded}
\begin{Highlighting}[]
\NormalTok{System.}\FunctionTok{out}\NormalTok{.}\FunctionTok{println}\NormalTok{(}\StringTok{"Das Resultat ist:"} \NormalTok{+ ganzeZahl);}
\end{Highlighting}
\end{Shaded}

\subsection{Eingabe}\label{eingabe}

Oft möchte man den Wert einer Variablen durch die Benutzerin oder den
Benutzer eines Programms bestimmen lassen. Eine einfache Möglichkeit ist
die Eingabe mit der Tastatur über das Konsolenfenster.

Eine Benutzereingabe ist in Java etwas aufwändiger als bei anderen
Programmiersprachen. Sie beinhaltet folgende zwei Schritte:

\textbf{Paket einbinden:} Mit einer Importanweisung zu Beginn unseres
Java-Programms muss zunächst die Klasse Scanner des Pakets util
eingebunden werden:

\begin{Shaded}
\begin{Highlighting}[]
\KeywordTok{import java.util.Scanner;}
\end{Highlighting}
\end{Shaded}

\textbf{Werte einlesen:} Mit folgenden zwei Zeilen können wir Werte in
Form von Zeichenketten (String) vom Konsolenfenster einlesen und einer
Variablen (z.B. wert) zuweisen:

\begin{Shaded}
\begin{Highlighting}[]
\NormalTok{Scanner eingabe = }\KeywordTok{new} \NormalTok{Scanner(System.}\FunctionTok{in}\NormalTok{);}
\NormalTok{String wert = eingabe.}\FunctionTok{next}\NormalTok{();}
\end{Highlighting}
\end{Shaded}

Nun weisen wir den eingelesenen Wert unserer Variablen \texttt{zahl} zu.
Hierfür muss der eingelesene Text noch in einen Integer umgewandelt
werden:

\begin{Shaded}
\begin{Highlighting}[]
\NormalTok{Integer.}\FunctionTok{parseInt}\NormalTok{(wert);}
\end{Highlighting}
\end{Shaded}

\textbf{Beispiel:} Mit den folgenden Anweisungen übergeben wir eine
Eingabezahl von der Konsole an die Variable \texttt{x} vom Typ Integer:

\begin{Shaded}
\begin{Highlighting}[]
\DataTypeTok{int} \NormalTok{x;}
\NormalTok{Scanner eingabe = }\KeywordTok{new} \NormalTok{Scanner(System.}\FunctionTok{in}\NormalTok{);}
\NormalTok{String wert = eingabe.}\FunctionTok{next}\NormalTok{();}
\NormalTok{x = Integer.}\FunctionTok{parseInt}\NormalTok{(wert);}
\end{Highlighting}
\end{Shaded}

\subsubsection{Einlesen von Datentypen}\label{einlesen-von-datentypen}

Für das Einlesen von den Standard Datentypen (siehe Tabelle
\ref{tab:datatypes}) bietet Scanner auch Möglichkeiten, diese direkt
einzulesen.

\paragraph{Beispiel:}\label{beispiel-9}

\begin{Shaded}
\begin{Highlighting}[]
\DataTypeTok{int} \NormalTok{ganzeZahl;}
\DataTypeTok{double} \NormalTok{kommaZahl;}
\NormalTok{ganzeZahl= eingabe.}\FunctionTok{nextInt}\NormalTok{();}
\NormalTok{kommaZahl= eingabe.}\FunctionTok{nextDouble}\NormalTok{();}
\end{Highlighting}
\end{Shaded}

\clearpage \begingroup \let\clearpage\relax \let\cleardoublepage\relax \chapter{ Theorieteil } \endgroup 

\section{Modulübersicht Test 2}\label{modulubersicht-test-2-1}

Die beiden Konzepte Variablen und Datentypen sind für jede
Programmierung grundlegend. Bei \textbf{Variablen} handelt es sich um
Speicherbereiche, in denen Werte gespeichert werden können und der
\textbf{Datentyp} gibt an, welche Werte erlaubt sind (z.B. nur
Ganzzahlen). In einem Programm werden Daten verarbeitet, die sich in
ihrer Art unterscheiden, z.B. Zeichen, Zahlen oder logische Daten.
Digitale Daten werden immer durch Ziffern dargestellt. Daher auch der
Name, \emph{digit} bedeutet Ziffer.

\section{Darstellung von Zahlen und Zeichen im
Computer}\label{darstellung-von-zahlen-und-zeichen-im-computer-1}

Um die Darstellung von Zeichen, Zahlen und Texten im Computer zu
verstehen, muss man das \textbf{binäre System} verstehen.

\subsection{Binäres System}\label{binares-system-1}

Alle Rechner stellen Information im binären System dar. Dieses kennt nur
zwei Ziffern, nämlich 0 und 1 (im Gegensatz zum Dezimalsystem mit den
Ziffern 0 bis 9). Eine solche Ziffer wird als \textbf{Bit} bezeichnet
(Abkürzung für \emph{Binary Digit}, übersetzt „Binäre Ziffer``). Ein Bit
stellt den kleinsten speicherbaren Wert in einem Computer dar. Jeweils 8
Bits werden zu einem \textbf{Byte} zusammengefasst. Ein Byte kann somit
2\textsuperscript{8} = 256 verschiedene Sequenzen von je 8 Bit
speichern.

\subsection{Darstellung von Zahlen im binären
System}\label{darstellung-von-zahlen-im-binaren-system-1}

Betrachten wir die Zahl 91, die binär mit 8 Bit als 01011011 dargestellt
wird (siehe Tabelle \ref{tab:binaryConversation}). Wir reden deswegen in
diesem Zusammenhang von der \textbf{Binärdarstellung} von 91 (und nicht
von der Dezimaldarstellung, die für uns lesefreundlicher ist).

\begin{table}[!htpb]
\begin{tabular}{|l|p{.8cm}|p{.8cm}|p{.8cm}|p{.8cm}|p{.7cm}|p{.7cm}|p{.7cm}|p{.7cm}|p{.8cm}}
\cline{1-9} \textbf{Bit} & 8 & 7 & 6 & 5 & 4 & 3 & 3 & 1 &\\
\cline{1-9} \textbf{Binärwert} & 0 & 1 & 0 & 1 & 1 & 0 & 1 & 1 &\\
\cline{1-9} \textbf{Wertigkeit} & $2^7=128$ & $2^6=64$ & $2^5=32$ & $2^4=16$ & $2^3=8$ & $2^2=4$ & $2^1=2$ & $2^0=1$ &\\
\hline \textbf{Dezimalwert} & 0 & 64 & 0 & 16 & 8 & 0 & 2 & 1 & \multicolumn{1}{ p{.8cm}| }{= 91} \\
\hline
\end{tabular}
\caption{Binäre Darstellung der Dezimalzahl 91. Details siehe Text.}
\label{tab:binaryConversation}
\end{table}

Eine 8-Bit-Zahl, wie in unserem Beispiel, kann Werte zwischen 00000000
(0 im Dezimalsystem) und 11111111 (255 im Dezimalsystem) speichern. Für
die Umrechnung vom Binär- in den Dezimalwert multiplizieren wir für
jedes Bit den Binärwert mit der Wertigkeit des Bits und summieren diese
auf. Im binären System können wir mit 8 Bit nur die ganzen Zahlen 0 bis
255 darstellen. Ist die Zahl, die wir darstellen wollen, grösser, muss
auch ein grösserer Speicherbereich bereitgestellt werden.

\subsection{Darstellung von Zeichen im binären
System}\label{darstellung-von-zeichen-im-binaren-system-1}

Für die Darstellung von Zeichen im Computer wurde der so genannte
\textbf{ASCII-Code} entwickelt. ASCII steht für \emph{American Standard
Code for Information Interchange}, was übersetzt so viel heisst wie
Amerikanische Standardcodierung für den Datenaustausch. Mit Hilfe des
7-Bit-ASCII-Codes können 128 verschiedene Zeichen (2\textsuperscript{7})
dargestellt werden oder umgekehrt wird jedem Zeichen ein Bitmuster aus 7
Bit zugeordnet (siehe Tabelle \ref{tab:ascii}). Die Zeichen entsprechen
weitgehend einer Computertastatur. Der ASCII-Code wurde später auf 8 Bit
erweitert, was die Darstellung von 256 Zeichen (2\textsuperscript{8})
erlaubt.

\begin{table}[!htbp]\centering
\ra{1.1}
\begin{tabular}{@{}llcllcllcll@{}}
\toprule
\multicolumn{2}{c}{0-31} & & \multicolumn{2}{c}{31-63} & &
\multicolumn{2}{c}{64-95} & & \multicolumn{2}{c}{96-127}\\
\cmidrule{1-2} \cmidrule{4-5} \cmidrule{7-8} \cmidrule{10-11}
Dez & Zeichen & & Dez  & Zeichen & & Dez & Zeichen & & Dez  & Zeichen\\
\midrule
0  & NUL & & 32 &  SP & & 64 & @ & & 96  & ` \\
1  & SOH & & 33 &  !  & & 65 & A & & 97  & a \\
2  & STX & & 34 &  "' & & 66 & B & & 98  & b \\
3  & ETX & & 35 &  \# & & 67 & C & & 99  & c \\
4  & EOT & & 36 &  \$ & & 68 & D & & 100 & d \\
5  & ENQ & & 37 &  \% & & 69 & E & & 101 & e \\
6  & ACK & & 38 &  \& & & 70 & F & & 102 & f \\
7  & BEL & & 39 &  '  & & 71 & G & & 103 & g \\
8  & BS  & & 40 &  (  & & 72 & H & & 104 & h \\
9  & HT  & & 41 &  )  & & 73 & I & & 105 & i \\
10 & LF  & & 42 &  *  & & 74 & J & & 106 & j \\
11 & VT  & & 43 &  +  & & 75 & K & & 107 & k \\
12 & FF  & & 44 &  ,  & & 76 & L & & 108 & l \\
13 & CR  & & 45 &  -  & & 77 & M & & 109 & m \\
14 & SO  & & 46 &  .  & & 78 & N & & 110 & n \\
15 & SI  & & 47 &  /  & & 79 & O & & 111 & o \\
16 & DLE & & 48 &  0  & & 80 & P & & 112 & p \\
17 & DC1 & & 49 &  1  & & 81 & Q & & 113 & q \\
18 & DC2 & & 50 &  2  & & 82 & R & & 114 & r \\
19 & DC3 & & 51 &  3  & & 83 & S & & 115 & s \\
20 & DC4 & & 52 &  4  & & 84 & T & & 116 & t \\
21 & NAK & & 53 &  5  & & 85 & U & & 117 & u \\
22 & SYN & & 54 &  6  & & 86 & V & & 118 & v \\
23 & ETB & & 55 &  7  & & 87 & W & & 119 & w \\
24 & CAN & & 56 &  8  & & 88 & X & & 120 & x \\
25 & EM  & & 57 &  9  & & 89 & Y & & 121 & y \\
26 & SUB & & 58 &  :  & & 90 & Z & & 122 & z \\
27 & ESC & & 59 &  ;  & & 91 & [ & & 123 & \{\\
28 & FS  & & 60 &  <  & & 92 & $\backslash$ & & 124 & $\mid$\\
29 & GS  & & 61 &  =  & & 93 & ]    & & 125 & \}  \\
30 & RS  & & 62 &  >  & & 94 & \^{} & & 126 & \textasciitilde \\
31 & US  & & 63 &  ?  & & 95 & \_   & & 127 & DEL \\
\bottomrule
\end{tabular}
\caption {ASCII-Tabelle}
\label{tab:ascii}
\end{table}

Die ASCII-Tabelle enthält auch nicht darstellbare Zeichen (wie etwa ein
Zeichen, das einen Zeilenumbruch repräsentiert). Die wichtigsten sind in
Tabelle \ref{tab:asciiHidden} dargestellt:

\begin{table}[!htbp]\centering
\begin{tabular}{lll}
\toprule
Dez & Zeichen & Bedeutung\\
\midrule
8   & BS & Backspace. Linkes Zeichen löschen\\
10  & NL & New Line. Neue Zeile beginnen\\
32  & SP & Space. Leerzeichen \\
127 & DEL & Delete. Rechtes Zeichen löschen \\
\bottomrule
\end{tabular}
\label{tab:asciiHidden}
\caption{Nicht darstellbare Zeichen der ASCII-Tabelle}
\end{table}

\section{Datentypen}\label{datentypen-1}

Der \textbf{Datentyp} gibt an, welche Daten in einem Programm
gespeichert und bearbeitet werden können. Programmiersprachen besitzen
vordefinierte Datentypen, die sich in der Art der Interpretation der
gespeicherten Daten und in der Grösse unterscheiden.

\begin{itemize}
\tightlist
\item
  Typ für Zahlenwerte
\item
  Typ für Zeichenwerte
\item
  Typ für Wahrheitswerte (Boolsche Werte) (siehe Modul 2)
\end{itemize}

Tabelle \ref{tab:datatypes} gibt einen Überblick über die wichtigsten
Datentypen, die in vielen Programmiersprachen vorkommen.

\begin{table}[!htbp]\centering 
\begin{tabular}{@{}llll@{}}     
\toprule
 Typ & Beschreibung & Grösse & Wertebereich \\
 & & in Bit & \\
\midrule
 boolean & Boolscher Wert & 1 & true oder false \\
 char & Zeichen & 16 & Unicode-Zeichen \\
 byte & Ganzzahl & 8 & $-128 \ldots 127 \;(-2^7 \ldots +2^{7}-1)$ \\
 short & & 16 & $-32'768 \ldots  32'767 \;(-2^{15} \ldots +2^{15}-1)$ \\
 int & & 32 & $-2'147'483'648 \ldots2'147'483'647 \;(-2^{31} \ldots +2^{31}-1)$ \\
 long & & 64 & $-9'223'372'036'854'775'808\ldots$ \\
 & & &  $9'223'372'036'854'775'807 \; (-2^{63} \ldots +2^{63}-1)$ \\
 float & Gleitkommazahl & 32 & $+/- 3.40282347 \times 10^{38}$\\
 double & & 64 & $+/- 1.79769313486231569 \times 10^{308}$ \\
\bottomrule

\end{tabular}
\caption{Die wichtigsten Datentypen in Java}
\label{tab:datatypes}
\end{table}

\section{Variablen und Konstanten}\label{variablen-und-konstanten-1}

\textbf{Variablen} einer Programmiersprache sind denen der Mathematik
ähnlich. Variablen haben einen Namen über den sie angesprochen werden,
und dienen dazu, einen Wert im Bereich eines bestimmten Datentyps zu
speichern. Der Wert der Variable kann sich im Verlaufe des Programms
ändern (er kann variieren, daher der Name). Um eine Variable in einem
Programm verwenden zu können, sind folgende Operationen notwendig:

\begin{enumerate}
\def\labelenumi{\arabic{enumi}.}
\tightlist
\item
  Deklaration
\item
  Initialisierung
\end{enumerate}

\subsection{Deklaration}\label{deklaration-1}

Bevor eine Variable in einem Programm verwendet werden kann, muss sie
\textbf{deklariert} werden. Das heisst, dass Sie als Programmiererin
oder Programmierer einen Speicherbereich für einen bestimmten Datentyp
belegen und diesem Speicherplatz einen Namen geben. Über diesen Namen
kann der Speicherbereich während des Programmablaufs aufgerufen werden.
Namen von Variablen beginnen in Java gemäss Konvention jeweils mit einem
Kleinbuchstaben, sie dürfen keine Leerzeichen enthalten und sollten
möglichst aussagekräftig sein.

\paragraph{Schreibweise:}\label{schreibweise-3}

\begin{verbatim}
Datentyp name;
\end{verbatim}

\paragraph{Beispiel:}\label{beispiel-10}

\begin{Shaded}
\begin{Highlighting}[]
\CommentTok{// Variable a vom Typ Integer}
\DataTypeTok{int} \NormalTok{a;}

\CommentTok{// Variable b vom Typ Double}
\DataTypeTok{double} \NormalTok{b;}

\CommentTok{// Variable c vom Typ Character}
\DataTypeTok{char} \NormalTok{c;}
\end{Highlighting}
\end{Shaded}

Mehrere Variablen vom gleichen Typ können auch wie in folgendem Beispiel
in einer Deklaration geschrieben werden:

\begin{Shaded}
\begin{Highlighting}[]
\CommentTok{// 3 Variablem vom Typ Integer}
\DataTypeTok{int} \NormalTok{meineZahl1, meineZahl2, meineZahl3;}
\end{Highlighting}
\end{Shaded}

\subsection{Initialisierung und
Wertzuweisung}\label{initialisierung-und-wertzuweisung-1}

Das \textbf{Zuweisen} von Werten geschieht mit dem Zuweisungsoperator.
In Java wird hierfür ein \textbf{Gleichheitszeichen} (\textbf{=})
verwendet. Dabei wird der Wert des Ausdrucks rechts des
Zuweisungsoperators der Variablen auf der linken Seite zugewiesen. Wenn
einer Variable das erste Mal ein Wert zugewiesen wird, spricht man von
einer \textbf{Initialisierung}.

\paragraph{Schreibweise:}\label{schreibweise-4}

\begin{Shaded}
\begin{Highlighting}[]
\NormalTok{variable = wert;}
\end{Highlighting}
\end{Shaded}

\paragraph{Beispiel:}\label{beispiel-11}

\begin{Shaded}
\begin{Highlighting}[]
\NormalTok{meineZahl = }\DecValTok{4}\NormalTok{;}
\CommentTok{// meineZahl hat den Wert 4}
\end{Highlighting}
\end{Shaded}

Der Wert einer Variablen kann sich im Verlaufe eines Programms ändern.
In folgendem Beispiel wird in der Variablen \texttt{meineZahl} zuerst
der Wert 4 gespeichert, der dann in einer weiteren Zeile mit dem Wert 6
überchrieben wird.

\begin{Shaded}
\begin{Highlighting}[]
\NormalTok{meineZahl = }\DecValTok{4}\NormalTok{;}
\CommentTok{// Wert von meineZahl ist 4}

\NormalTok{meineZahl = }\DecValTok{6}\NormalTok{;}
\CommentTok{// Wert von meineZahl ist 6}
\end{Highlighting}
\end{Shaded}

Bei einer Zuweisung handelt es sich also immer um einen schreibenden
Zugriff auf eine Variable mit dem Resultat, dass sich deren Wert ändern
kann. Der alte Wert wird überschrieben. Damit einer Variablen ein Wert
zugewiesen werden kann, darf die Variable nicht als Konstante definiert
sein (siehe nächster Abschnitt) und der Typ der Variablen muss mit dem
Typ des Werts kompatibel sein. Auf jeden Fall kompatibel sind Variablen
und Werte desselben Datentypes. Wenn die Datentypen nicht
übereinstimmen, nimmt Java eine implizite \textbf{Typkonvertierung} vor.
Dies ist jedoch eine häufige Fehlerquelle und sollte daher vermieden
werden. Bei der impliziten Typkonvertierung in Java werden nur
Typkonvertierung durchgeführt, wenn sie ohne Informationsverlust
erfolgen kann, also wenn der Zieldatentyp einen gleichen oder grösseren
Wertebereich hat als der Ausgangsdatentyp.

\paragraph{Beispiel:}\label{beispiel-12}

Ein Wert vom Typ \texttt{int} kann einer Variablen vom Typ
\texttt{double} zugewiesen werden:

\begin{Shaded}
\begin{Highlighting}[]
\NormalTok{# rear}
\CommentTok{// Variable ganzeZahl vom Typ Integer}
\DataTypeTok{int} \NormalTok{ganzeZahl;}
\CommentTok{// Variable kommaZahl vom Typ Double}
\DataTypeTok{double} \NormalTok{kommaZahl;}

\NormalTok{ganzeZahl = }\DecValTok{4}\NormalTok{;}
\NormalTok{kommaZahl = ganzeZahl;}
\CommentTok{// Variable kommaZahl wird in zum Typ Integer konvertiert}
\end{Highlighting}
\end{Shaded}

Möchte man eine Zuweisung machen, bei der der Zieldatentyp einen
gleineren Wertebereich hat, muss eine so genannte explizite
Typenkonvertierung durchgeführt werden, das sogenannte Typecasting. Die
Programmiererin/der Programmierer ist dabei selber dafür verantwortlich,
dass die Zuweisung möglich ist.

\paragraph{Beispiel:}\label{beispiel-13}

\begin{Shaded}
\begin{Highlighting}[]
\CommentTok{// Variable d vom Typ Double}
\DataTypeTok{double} \NormalTok{d = }\FloatTok{1.3}\NormalTok{;}

\DataTypeTok{float} \NormalTok{f = (}\DataTypeTok{float}\NormalTok{)d;}
\CommentTok{// Variable d muss explizit zum Typ Float konvertiert }
\CommentTok{// werden}
\end{Highlighting}
\end{Shaded}

Eine Variable kann in einem Programm nur in einem bestimmten Bereich des
Programms gelten. Weiteres dazu erfahren Sie in einem späteren Modul.

\subsection{Konstanten}\label{konstanten-1}

\textbf{Konstanten} werden wie Variablen mit einem Namen bezeichnet. Sie
enthalten während der gesamten Programmausführung einen konstanten Wert.
Es kann also nach der Initialisierung keine weitere Wertzuweisung
vorgenommen werden. Konstanten können jedoch Teil einer Wertzuweisung an
Variablen sein. Eine Konstante wird zusätzlich zu Namen und Datentyp mit
dem Schlüsselwort \texttt{final} deklariert.

\paragraph{Schreibweise:}\label{schreibweise-5}

\begin{Shaded}
\begin{Highlighting}[]
\DataTypeTok{final} \NormalTok{Datentyp name;}
\end{Highlighting}
\end{Shaded}

\paragraph{Beispiel:}\label{beispiel-14}

\begin{Shaded}
\begin{Highlighting}[]
\CommentTok{//Deklaration und Initialisierung der Konstante k}
\DataTypeTok{final} \DataTypeTok{int} \NormalTok{k = }\DecValTok{4}\NormalTok{;}
\end{Highlighting}
\end{Shaded}

\section{Operatoren und Ausdrücke}\label{operatoren-und-ausdrucke-1}

\subsection{Operatoren (Teil I)}\label{operatoren-teil-i-1}

Um in einem Programm Berechnungen durchzuführen zu können, stehen
folgende \textbf{arithmetische Operatoren} zur Verfügung:

\begin{table}[!htpb]\centering
\begin{tabular}{@{}llll@{}}

\toprule
 Operator & Ausdruck & Beschreibung & Liefert \\
\midrule
 + & a + b & Addition & Summe \\
 - & a - b& Subtraktion & Differenz \\
 $\ast$ & a $\ast$ b& Multiplikation& Produkt \\
 / & a / b& Division& Quotient \\
 \% & a \% b& Modulo& Ganzzahliger Rest einer Division \\
\bottomrule
\end{tabular}
\caption{Arithmetische Operatoren in Java}
\label{tab:Operators1}
\end{table}

Weitere Operatoren (logische und Vergleichsoperatoren) lernen Sie in
Modul 2 kennen.

\subsection{Ausdrücke}\label{ausdrucke-1}

\textbf{Ausdrücke} (engl. \emph{expressions}) sind in einer
Programmiersprache Teil der kleinsten ausführbaren Einheiten eines
Programms. Dabei handelt es sich um Verarbeitungsvorschriften, die sich
aus \textbf{Variablen}, \textbf{Konstanten} und \textbf{Operatoren}
zusammensetzen können und ein Resultat ergeben. Variablen und
Konstanten, die mit einem Operator verknüpft werden, nennt man
\textbf{Operanden}. Ein Ausdruck kann auch aus einer einzelnen Variablen
bestehen.

Folgendes Beispiel zeigt einen Ausdruck, der aus einer Variablen
\texttt{i}, einem Operator \texttt{+} und einem Operanden \texttt{5}
besteht.

\begin{Shaded}
\begin{Highlighting}[]
  \NormalTok{i + }\DecValTok{5}
\end{Highlighting}
\end{Shaded}

Das Resultat des Ausdrucks kann wieder in einer Variablen gespeichert
werden. In folgendem Beispiel wird das Resultat in der Variablen
\texttt{i} gespeichert. Der vorherige Wert von \texttt{i} wird dadurch
überschrieben.

\begin{Shaded}
\begin{Highlighting}[]
  \NormalTok{i = i + }\DecValTok{5}\NormalTok{;}
\end{Highlighting}
\end{Shaded}

Die Reihenfolge, mit der Ausdrücke bearbeitet werden, kann durch die
Wahl des Operators und durch Klammern beeinflusst werden. Hierfür gelten
die mathematischen Regeln, wie wir sie in der Schule gelernt haben, also
„Klammern zuerst, dann Punkt vor Strich``.

\paragraph{Beispiel:}\label{beispiel-15}

\begin{Shaded}
\begin{Highlighting}[]
  \DecValTok{5} \NormalTok{* (}\DecValTok{2} \NormalTok{+ }\DecValTok{10}\NormalTok{)}
\end{Highlighting}
\end{Shaded}

Die Klammern erzwingen, dass die Addition vor der Multiplikation
ausgeführt wird.

\subsection{Weitere Arithmetische
Operatoren}\label{weitere-arithmetische-operatoren-1}

Es gibt in Java noch weitere arithmetische Operatoren.

\paragraph{Zuweisungsoperator:}\label{zuweisungsoperator-1}

\begin{Shaded}
\begin{Highlighting}[]
\NormalTok{i += }\DecValTok{1}\NormalTok{; \textbackslash{}\textbackslash{}entspricht i = i + }\DecValTok{1}\NormalTok{;}
\NormalTok{i -= }\DecValTok{1}\NormalTok{; \textbackslash{}\textbackslash{}entspricht i = i - }\DecValTok{1}\NormalTok{;}
\NormalTok{i *= }\DecValTok{1}\NormalTok{; \textbackslash{}\textbackslash{}entspricht i = i * }\DecValTok{1}\NormalTok{;}
\NormalTok{i /= }\DecValTok{1}\NormalTok{; \textbackslash{}\textbackslash{}entspricht i = i / }\DecValTok{1}\NormalTok{;}
\NormalTok{i %= }\DecValTok{1}\NormalTok{; \textbackslash{}\textbackslash{}entspricht i = i % }\DecValTok{1}\NormalTok{;}
\end{Highlighting}
\end{Shaded}

Die Zuweisungsoperatoren dienen dazu die Anweisungen kompakter
darzustellen, da man weniger Zeichen benötigt.

\paragraph{Increment und Decrement
Operatoren:}\label{increment-und-decrement-operatoren-1}

\begin{Shaded}
\begin{Highlighting}[]
\NormalTok{i++; \textbackslash{}\textbackslash{}entspricht i = i + }\DecValTok{1}\NormalTok{;}
\NormalTok{i--; \textbackslash{}\textbackslash{}entspricht i = i - }\DecValTok{1}\NormalTok{;}
\end{Highlighting}
\end{Shaded}

Diese Operatoren sind meistens in sogenannten for-Schleifen anzutreffen,
wo sie einen Zähler hochzählen (siehe Modul 2) .

\section{Der Datentyp String}\label{der-datentyp-string-1}

Der Datentyp \textbf{String} unterscheidet sich von den bisher
thematisierten Datentypen insofern, dass er eine Zusammenfassung von
mehreren gleichartigen Variablen darstellt. Dieser Datentyp ist auch
kein primitiver Datentyp mehr, da er mehrere Elemente zusammenfasst. Er
speichert nämlich alle Buchstaben einzeln in je einer char-Variablen.
Wie diese Zusammenfassung der einzelnen Buchstaben funktioniert, lernen
Sie, wenn es um die Objektorientierung geht. Die Deklaration und
Initialisierung der Variablen funktioniert jedoch wie in 1.4.1 und 1.4.2
beschreiben.

Bei der Initialisierung von String-Variablen muss der Wert zwischen
\textbf{doppelten Hochkommata} (``) angegeben werden.

\paragraph{Beispiel:}\label{beispiel-16}

\begin{Shaded}
\begin{Highlighting}[]
\CommentTok{// Deklaration des Strings vorname}
\NormalTok{String vorname;}

\CommentTok{// Initialisierung mit dem Wert "Paul"}
\NormalTok{vorname = }\StringTok{"Paul"}\NormalTok{;}
\end{Highlighting}
\end{Shaded}

Da ein String mehrere char-Variablen enthält, kann dem String auch ein
einzelner char zugewiesen werden.

\paragraph{Beispiel:}\label{beispiel-17}

\begin{Shaded}
\begin{Highlighting}[]
\NormalTok{String name;}
\NormalTok{name = 'a';}
\end{Highlighting}
\end{Shaded}

Mehrere Strings können mit einem \textbf{Plus} (+) verbunden werden. So
entsteht aus mehreren Einzelteilen ein neuer Text.

\paragraph{Beispiel:}\label{beispiel-18}

\begin{Shaded}
\begin{Highlighting}[]
\NormalTok{String text;}
\NormalTok{text = }\StringTok{"Hallo, "}\NormalTok{+ }\StringTok{"das "} \NormalTok{+ }\StringTok{"sind "} \NormalTok{+ }\StringTok{"mehrere "} \NormalTok{+ }\StringTok{"Wörter."}\NormalTok{;}
\end{Highlighting}
\end{Shaded}

\section{Bildschirm- Ein- und
Ausgabe}\label{bildschirm--ein--und-ausgabe-1}

Oft möchte man, dass die Benutzerin oder der Benutzer des Programms mit
dem Programm interagieren kann. Das bedeutet, dass das Programm eine
Ausgabe macht oder dass die Benutzerin oder der Benutzer etwas eingeben
kann. Um dies zu realisieren verwenden wir Funktionalitäten, welche von
Java zur Verfügung gestellt werden.

\subsection{Ausgabe}\label{ausgabe-1}

Damit die Benutzerin oder der Benutzer auch sieht, was im Programm
berechnet wurde, kann im Programmcode angegeben werden, dass ein
bestimmter Text oder der Wert einer Variablen ausgegeben wird.

\paragraph{Beispiel: Ausgabe eines vorgegebenen
Texts}\label{beispiel-ausgabe-eines-vorgegebenen-texts-1}

\begin{Shaded}
\begin{Highlighting}[]
\NormalTok{System.}\FunctionTok{out}\NormalTok{.}\FunctionTok{println}\NormalTok{(}\StringTok{"Das Programm hat geendet."}\NormalTok{);}
\end{Highlighting}
\end{Shaded}

Im obigen Beispiel wird der Text „Das Programm hat geendet.''
ausgegeben. Der Text, der ausgegeben wird, steht zwischen einem Paar von
Anführungs- und Schlusszeichen (``), die nicht mit ausgegeben werden.
Man möchte aber nicht immer nur vorgegebenen Text ausgeben, sondern z.B.
das Resultat einer Berechnung, welches in einer Variablen (z.B.
\texttt{ganzeZahl}) gespeichert ist.

\paragraph{Beispiel: Ausgabe des Wertes einer
Variablen}\label{beispiel-ausgabe-des-wertes-einer-variablen-1}

\begin{Shaded}
\begin{Highlighting}[]
\NormalTok{System.}\FunctionTok{out}\NormalTok{.}\FunctionTok{println}\NormalTok{(ganzeZahl);}
\end{Highlighting}
\end{Shaded}

Wenn man Variablenwerte und Text verbinden möchte, geschieht dies mit
einem Additions-Zeichen (\texttt{+}).

\paragraph{Beispiel: Ausgabe von Text und
Variablenwert}\label{beispiel-ausgabe-von-text-und-variablenwert-1}

\begin{Shaded}
\begin{Highlighting}[]
\NormalTok{System.}\FunctionTok{out}\NormalTok{.}\FunctionTok{println}\NormalTok{(}\StringTok{"Das Resultat ist:"} \NormalTok{+ ganzeZahl);}
\end{Highlighting}
\end{Shaded}

\subsection{Eingabe}\label{eingabe-1}

Oft möchte man den Wert einer Variablen durch die Benutzerin oder den
Benutzer eines Programms bestimmen lassen. Eine einfache Möglichkeit ist
die Eingabe mit der Tastatur über das Konsolenfenster.

Eine Benutzereingabe ist in Java etwas aufwändiger als bei anderen
Programmiersprachen. Sie beinhaltet folgende zwei Schritte:

\textbf{Paket einbinden:} Mit einer Importanweisung zu Beginn unseres
Java-Programms muss zunächst die Klasse Scanner des Pakets util
eingebunden werden:

\begin{Shaded}
\begin{Highlighting}[]
\KeywordTok{import java.util.Scanner;}
\end{Highlighting}
\end{Shaded}

\textbf{Werte einlesen:} Mit folgenden zwei Zeilen können wir Werte in
Form von Zeichenketten (String) vom Konsolenfenster einlesen und einer
Variablen (z.B. wert) zuweisen:

\begin{Shaded}
\begin{Highlighting}[]
\NormalTok{Scanner eingabe = }\KeywordTok{new} \NormalTok{Scanner(System.}\FunctionTok{in}\NormalTok{);}
\NormalTok{String wert = eingabe.}\FunctionTok{next}\NormalTok{();}
\end{Highlighting}
\end{Shaded}

Nun weisen wir den eingelesenen Wert unserer Variablen \texttt{zahl} zu.
Hierfür muss der eingelesene Text noch in einen Integer umgewandelt
werden:

\begin{Shaded}
\begin{Highlighting}[]
\NormalTok{Integer.}\FunctionTok{parseInt}\NormalTok{(wert);}
\end{Highlighting}
\end{Shaded}

\textbf{Beispiel:} Mit den folgenden Anweisungen übergeben wir eine
Eingabezahl von der Konsole an die Variable \texttt{x} vom Typ Integer:

\begin{Shaded}
\begin{Highlighting}[]
\DataTypeTok{int} \NormalTok{x;}
\NormalTok{Scanner eingabe = }\KeywordTok{new} \NormalTok{Scanner(System.}\FunctionTok{in}\NormalTok{);}
\NormalTok{String wert = eingabe.}\FunctionTok{next}\NormalTok{();}
\NormalTok{x = Integer.}\FunctionTok{parseInt}\NormalTok{(wert);}
\end{Highlighting}
\end{Shaded}

\subsubsection{Einlesen von Datentypen}\label{einlesen-von-datentypen-1}

Für das Einlesen von den Standard Datentypen (siehe Tabelle
\ref{tab:datatypes}) bietet Scanner auch Möglichkeiten, diese direkt
einzulesen.

\paragraph{Beispiel:}\label{beispiel-19}

\begin{Shaded}
\begin{Highlighting}[]
\DataTypeTok{int} \NormalTok{ganzeZahl;}
\DataTypeTok{double} \NormalTok{kommaZahl;}
\NormalTok{ganzeZahl= eingabe.}\FunctionTok{nextInt}\NormalTok{();}
\NormalTok{kommaZahl= eingabe.}\FunctionTok{nextDouble}\NormalTok{();}
\end{Highlighting}
\end{Shaded}


\end{document}

